\section{Methodology}

This chapter outlines the methodological approach used for designing, developing, and evaluating the subscription-based Learning Management System.

\subsection{Research Approach}

This project employed a design science research methodology as described by Hevner \cite{hevner2004}, which is appropriate for creating innovative information systems artifacts. The process followed six key steps: problem identification, objective definition, design and development, demonstration, evaluation, and communication.

Requirements were gathered through:

\begin{itemize}
    \item \textbf{Literature Analysis:} Review of 42 academic papers and 15 industry reports (2014-2023) on LMS platforms, e-learning business models, and educational technology
    
    \item \textbf{Platform Analysis:} Competitive evaluation of traditional LMS platforms (Moodle, Canvas), course marketplaces (Udemy, Coursera), and creator-focused platforms (Teachable, Podia)
    
    \item \textbf{Secondary User Research:} Analysis of published surveys, user reviews, and forum discussions to provide user insights within project constraints
\end{itemize}

\subsection{System Design and Development}

\subsubsection{Design Approach}

The system was designed using an object-oriented approach with:

\begin{itemize}
    \item \textbf{MVC Architecture:} Selected for separation of data, presentation, and application logic \cite{leff2001}
    
    \item \textbf{Entity-Relationship Modeling:} Used for database schema design, identifying key entities, defining relationships, and implementing with SQLAlchemy ORM
    
    \item \textbf{User-Centered UI Design:} Focused on accessibility (WCAG 2.1), responsive design, consistent navigation, and progressive disclosure
\end{itemize}

\subsubsection{Development Process}

The development process followed agile methodology with iterative cycles \cite{dingsoyr2012}. The technology stack included:

\begin{itemize}
    \item Backend: Python with Flask framework
    \item Database: SQLite with SQLAlchemy ORM
    \item Frontend: HTML5, CSS3, JavaScript with Bootstrap
    \item Version Control: Git
\end{itemize}

Development was organized into phases: environment setup, database implementation, core functionality, subscription management, payment verification, content delivery, UI refinement, and testing.

\subsection{Testing and Evaluation}

Testing employed a multi-level approach:

\begin{itemize}
    \item \textbf{Unit Testing:} Individual component testing using Python's unittest framework
    
    \item \textbf{Integration Testing:} Verification of component interactions for user registration, course creation, enrollment, and content access
    
    \item \textbf{UI Testing:} Cross-browser and multi-device testing for responsive behavior
    
    \item \textbf{Performance Evaluation:} Assessment of response times, query efficiency, and resource utilization
    
    \item \textbf{Usability Evaluation:} Heuristic analysis based on Nielsen's principles \cite{nielsen1994} and task completion assessment
\end{itemize}

\subsection{Ethical Considerations and Limitations}

Ethical dimensions addressed included privacy (data minimization), accessibility (inclusive design), fairness (balanced pricing models), and transparency (clear fee communication).

Methodological limitations included restricted direct user research, simplified implementation of certain features, prioritization of functionality over optimization, and limited testing environments. Despite these constraints, the methodology provided a structured approach balancing academic requirements with practical implementation.
\section{System Design}

This chapter presents the design of the subscription-based Learning Management System, focusing on architecture, database design, and user interface considerations.

\subsection{System Architecture}

The system uses a Model-View-Controller (MVC) architectural pattern that separates the application into three interconnected components, enhancing maintainability and scalability \cite{krasner1988}.

\begin{figure}[h]
\centering
% You can insert a screenshot or diagram here
\caption{High-level System Architecture}
\label{fig:architecture}
\end{figure}

The architecture includes:

\begin{itemize}
    \item \textbf{Client Layer:} Browser-based interface using HTML, CSS, JavaScript with Bootstrap
    \item \textbf{Application Layer:} Flask-based web application implementing business logic and authentication
    \item \textbf{Data Layer:} SQLite database with SQLAlchemy ORM for data storage
    \item \textbf{File Storage:} Local storage for course materials and payment receipts
\end{itemize}

Security is implemented through multiple layers: authentication using Flask-Login with password hashing, role-based access control, CSRF protection, input validation, and secure file upload validation.

\subsection{Database Design}

The database design utilizes Entity-Relationship modeling to represent relationships between users, courses, subscriptions, and educational content.

\begin{figure}[h]
\centering
% You can insert a diagram here
\caption{Entity-Relationship Diagram}
\label{fig:erd}
\end{figure}

Key entities include:

\begin{itemize}
    \item \textbf{User:} System users with role designation (admin, teacher, student)
    \item \textbf{TeacherSubscriptionPlan:} Subscription tiers with student capacity limits
    \item \textbf{Course:} Educational courses with pricing and subscription information
    \item \textbf{Module/Material:} Course content organization and storage
    \item \textbf{Enrollment:} Student course access with payment verification
    \item \textbf{Assignment/Submission:} Student work and assessment records
    \item \textbf{Test/Question:} Assessment tools and student responses
\end{itemize}

\begin{lstlisting}[language=Python, caption=Course Model Example, label=lst:course-model]
class Course(db.Model):
    __tablename__ = 'courses'
    
    id = db.Column(db.Integer, primary_key=True)
    title = db.Column(db.String(100), nullable=False)
    description = db.Column(db.Text, nullable=False)
    
    # Payment related fields
    enrollment_price = db.Column(db.Integer, default=0)
    subscription_plan_id = db.Column(db.Integer, 
                              db.ForeignKey('teacher_subscription_plans.id'))
    payment_receipt_path = db.Column(db.String(255))
    payment_verified = db.Column(db.Boolean, default=False)
    
    # Relationships
    teacher_id = db.Column(db.Integer, db.ForeignKey('users.id'))
    teacher = db.relationship('User', backref='courses')
    modules = db.relationship('Module', backref='course',
                           cascade='all, delete-orphan')
\end{lstlisting}

\subsection{Subscription and Payment Model}

The system's key innovation is its subscription and payment model:

\subsubsection{Teacher Subscription Tiers}

\begin{table}[h]
\centering
\begin{tabular}{|l|c|c|l|}
\hline
\textbf{Plan Level} & \textbf{Max Students} & \textbf{Monthly Price (KZT)} & \textbf{Features} \\
\hline
Standard & 10 & 20,000 & Basic course tools \\
Pro & 30 & 35,000 & + Advanced analytics \\
Premium & 100 & 70,000 & + Priority support \\
Enterprise & Unlimited & 100,000 & Full feature set \\
\hline
\end{tabular}
\caption{Teacher Subscription Plans}
\label{tab:subscription-plans}
\end{table}

\subsubsection{Payment Verification Approach}

Rather than processing payments directly, the system implements external payment verification:

\begin{enumerate}
    \item Teachers upload proof of subscription payment
    \item Administrators verify before approving courses
    \item Students upload course payment receipts
    \item Teachers verify student payments for access
\end{enumerate}

This approach reduces regulatory complexity while maintaining financial integrity.

\subsection{User Interface Design}

The interface emphasizes consistency, clarity, efficiency, responsiveness, and accessibility. Key user flows include:

\textbf{Teacher Flow:} Register → Select subscription → Upload payment → Create courses → Manage content → Verify student enrollments

\textbf{Student Flow:} Register → Browse courses → Enroll with payment → Access materials → Complete assignments

\textbf{Administrator Flow:} Manage users → Verify payments → Approve courses → Monitor system activity

\begin{figure}[h]
\centering
% Insert screenshot or mockup here
\caption{Teacher Dashboard Interface}
\label{fig:teacher-dashboard}
\end{figure}

\subsection{Technology Stack}

The technology selection prioritized development efficiency, performance, and security:

\begin{itemize}
    \item \textbf{Backend:} Python with Flask (chosen for lightweight architecture and SQLAlchemy integration)
    \item \textbf{Database:} SQLite for development, with migration path to PostgreSQL
    \item \textbf{Frontend:} HTML5, CSS3, JavaScript, Bootstrap 4, and jQuery
\end{itemize}

\subsection{Design Challenges and Solutions}

Three key challenges were addressed during system design:

\begin{itemize}
    \item \textbf{Balancing Flexibility and Complexity:} Solved using modular design with progressive disclosure UI techniques
    
    \item \textbf{Payment Verification Security:} Implemented multi-layered verification with image validation, admin review, and audit trails
    
    \item \textbf{Access Control Granularity:} Created role-based access with payment verification, enforced at controller and view levels
\end{itemize}
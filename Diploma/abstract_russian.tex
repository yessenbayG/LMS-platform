\newpage
\section*{Аннотация}
\thispagestyle{empty}

В этом дипломном проекте представлены дизайн и реализация системы управления обучением (LMS) на основе подписки, которая предлагает платформу, где преподаватели могут публиковать образовательный контент на основе уровней подписки, а студенты могут покупать доступ к отдельным курсам. Система отвечает потребности в устойчивой образовательной платформе, где создатели контента получают вознаграждение за свои знания, а учащиеся имеют доступ к качественному образованию по доступным ценам.

Разработанная LMS включает такие ключевые функции, как многоуровневые планы подписки для учителей на основе ограничений вместимости учащихся, модели ценообразования курсов, безопасную проверку платежей, инструменты управления контентом и интерактивные учебные модули. Система использует современные веб-технологии, включая Python, Flask, SQLAlchemy и Bootstrap для адаптивного дизайна.

Методология исследования включала анализ требований, проектирование системы с использованием объектно-ориентированных принципов, гибкую реализацию и тестирование удобства использования. Оценка демонстрирует, что система эффективно управляет доставкой образовательного контента при безопасной обработке финансовых транзакций.

Платформа предлагает преимущества нескольким заинтересованным сторонам: учителя получают канал монетизации для своего образовательного контента, студенты получают доступ к качественным курсам по разным ценам, а образовательные учреждения могут использовать платформу для расширения своего охвата за пределы физических аудиторий.

\textbf{Ключевые слова:} Система управления обучением, Модель на основе подписки, Электронное обучение, Проверка платежей, Образовательная технология, Веб-разработка
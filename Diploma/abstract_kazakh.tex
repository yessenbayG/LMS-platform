\newpage
\section*{Аңдатпа}
\thispagestyle{empty}

Бұл дипломдық жоба оқытушылар жазылым деңгейлеріне негізделген білім беру мазмұнын жариялай алатын, ал студенттер жеке курстарға қол жетімділікті сатып ала алатын жазылымға негізделген Оқытуды басқару жүйесін (LMS) жобалау мен іске асыруды ұсынады. Жүйе мазмұн жасаушылар өз сараптамасы үшін сыйақы алатын, ал оқушылар қолжетімді бағамен сапалы білім алатын тұрақты білім беру платформасына деген қажеттілікті қанағаттандырады.

Әзірленген LMS студенттердің сыйымдылық шектеріне негізделген мұғалімдерге арналған сатылы жазылым жоспарлары, курстық баға модельдері, қауіпсіз төлемдерді тексеру, мазмұнды басқару құралдары және интерактивті оқыту модульдері сияқты негізгі мүмкіндіктерді қамтиды. Жүйе заманауи веб-технологияларды, соның ішінде Python, Flask, SQLAlchemy және жауап беретін дизайнға арналған Bootstrap қолданады.

Зерттеу әдістемесі талаптарды талдауды, объектіге бағытталған принциптерді қолдана отырып жүйені жобалауды, икемді енгізуді және қолдану ыңғайлылығын тексеруді қамтыды. Бағалау жүйенің қаржылық транзакцияларды қауіпсіз өңдеу арқылы білім беру мазмұнын тиімді басқаратынын көрсетеді.

Платформа бірнеше мүдделі тараптарға пайда әкеледі: мұғалімдер өздерінің білім беру мазмұнын монетизациялау арнасына ие болады, студенттер әртүрлі баға нүктелеріндегі сапалы курстарға қол жеткізеді, ал білім беру мекемелері физикалық аудиториялардан тыс өз ықпалын кеңейту үшін платформаны пайдалана алады.

\textbf{Кілт сөздер:} Оқытуды басқару жүйесі, Жазылымға негізделген модель, Электрондық оқыту, Төлемді растау, Білім беру технологиясы, Веб-әзірлеу